\documentclass{beamer}
\usepackage{graphicx} % Required for inserting images
\usepackage[dvipsnames]{xcolor}
\usepackage[italian]{babel}
\usepackage{ragged2e}
\usepackage{hyphenat}
\usepackage{biblatex}
\addbibresource{bib.bib}

\mode<presentation>{
  \usetheme{Frankfurt}
  \setbeamertemplate{headline}{
  \leavevmode%
  \begin{beamercolorbox}[wd=\paperwidth,ht=0.8pt]{frametitle right} % puoi usare anche un colore personalizzato qui
  \end{beamercolorbox}
}

%  Here is a gallery with other themes:
%  http://deic.uab.es/~iblanes/beamer_gallery/
  \usecolortheme[named=PineGreen]{structure}
\useoutertheme[footline=authortitle,subsection=false]{miniframes}
\makeatletter
\setbeamertemplate{footline}{
  \leavevmode%
  \hbox{%
    \begin{beamercolorbox}[wd=.65\paperwidth,ht=2.5ex,dp=1ex,center]{author in head/foot}%
      \usebeamerfont{author in head/foot}\insertshortauthor
    \end{beamercolorbox}%
    \begin{beamercolorbox}[wd=.1\paperwidth,ht=2.5ex,dp=1ex,center]{frame number in head/foot}%
      \insertframenumber{} / \inserttotalframenumber
    \end{beamercolorbox}%
    \begin{beamercolorbox}[wd=.25\paperwidth,ht=2.5ex,dp=1ex,center]{title in head/foot}%
      \usebeamerfont{title in head/foot}\insertshorttitle
    \end{beamercolorbox}%
  }
  \vskip0pt%
}
 	\setbeamercovered{transparent}
	\setbeamercolor{block title example}{fg=white,bg=Blue}
	\setbeamercolor{block body example}{fg=black,bg=Blue!10}
	\setbeamercolor{postit}{fg=black,bg=OliveGreen!20}
	\setbeamercolor{postit2}{fg=yellow,bg=OliveGreen}
%    \setbeamercolor{NEW_STYLE_NAME}{fg=COLOR_FOREGROUNG,bg=COLOR_BACKGROUNG}
}

\title[Computational Imaging]{Meteorological Super-Resolution\\vs\\Wind Representations}
\author[Gruppo 21 - Marzia De Maina, Matteo Galiazzo, Federica Santisi]
{Gruppo 21\\Marzia De Maina, Matteo Galiazzo, Federica Santisi}
\institute[Alma Mater Studiorum - Università di Bologna]
{
  \textit{Alma Mater Studiorum - Università di Bologna}\\[0.25Cm]
  \textit{Dipartimento di Informatica - Scienza e Ingegneria (DISI)} \\[0.5Cm]
  Prof. \textbf{Fabio Merizzi}\\
  }
\date{}
\begin{document}
\begin{frame}
    \titlepage
    
\end{frame}

\begin{frame}{Purpose}
\begin{center}
\justifying
The purpose of the project is to study the impact of different wind representations in the context of super-resolution.
\end{center}
\end{frame}
% COSE DA AGGIUNGERE PER ROSICCHIARE TEMPO
% CONFIFGURAZIONE DELL'AMBIENTE, DOVE ABBIAMO FATTO IL TRAINING E COME ABBIAMO GESTITO IL TUTTO

% MARZIA
\begin{frame}{Problem explaination and theory recap}
\cite{merizzi}
% riprendere la consegna ricordando che consegna avevamo, cosa era richiesto e fare una panoramica sulla teoria necessaria (unet)
\end{frame}

% FEDE
\begin{frame}{Datasets and Preprocessing}
% parlare dei 2 dataset (all'inizio erano 3) + come li abbiamo gestiti (preprocessing + abbiamo ritagliato per salvare risorse + minmax normalizzatione)
\end{frame}

% MATTEO
\begin{frame}{Model and hyperparameters}
% spiegare che modello abbiamo usato (upconv vs bilinear upsampling) + come abbiamo scelto la loss + come abbiamo scelto gli iperparametri
\end{frame}

% MATTEO
\begin{frame}{Training}
% training falliti + poi come ce la siamo cavata
\end{frame}

% FEDE
\begin{frame}{Results}
% mostrare i risultati finali per entrambi + un commento sulla differenza tra training locale con 100 immagini e training su colab con 1000
\end{frame}

% MARZIA
\begin{frame}{Final Considerations}
% come ci aspettavamo la rappresentazione con vettori e' quella migliore e quella con intensita' e direzione fa schifo al cazzo
\end{frame}

\begin{frame}
    \frametitle{References}
    \printbibliography
\end{frame}
\end{document}
